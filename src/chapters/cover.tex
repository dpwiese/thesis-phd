\title{Systematic Adaptive Control Design Using Sequential Loop Closure}
\author{Daniel Philip Wiese}
\prevdegrees{B.S., University of California, Davis (2011) \\
S.M., Massachusetts Institute of Technology (2013)}
\department{Department of Mechanical Engineering}
\degree{Doctor of Philosophy}
\degreemonth{June}
\degreeyear{2016}
\thesisdate{May 19, 2016}
\supervisor{Anuradha M. Annaswamy}{Senior Research Scientist}
\chairman{Rohan Abeyaratne}{Chairman, Department Committee on Graduate Students}
\maketitle

\cleardoublepage%
\setcounter{savepage}{\thepage}
\begin{abstractpage}
  This thesis presents a new, systematic method of synthesizing an output feedback adaptive controller for a class of uncertain, non-square multi-input/multi-output systems.
The control design process consists of first designing an inner-loop controller for a reduced order plant model to enforce command tracking of selected inner-loop variables, with an adaptive element used to accommodate parametric uncertainties in the plant.
Once this inner-loop control design is complete, an outer-loop is then designed which prescribes the inner-loop commands to enforce command tracking of selected outer-loop variables.

The main challenge that needs to be addressed when designing the inner-loop controller is the determination of a corresponding square and strictly positive real transfer function.
The first contribution of this thesis is the design of a new procedure to synthesize two gain matrices that allow the realization of such a transfer function, thereby allowing a globally stable adaptive output feedback law to be generated.
The unique features of this output feedback adaptive controller are a baseline controller that uses a Luenberger observer, a closed-loop reference model, manipulations of a bilinear matrix inequality, and the Kalman-Yakubovich lemma.
Using these features, a simple design procedure is proposed for the adaptive controller, and the corresponding stability property is established.

The outer-loop controller is designed around the plant with existing adaptive inner-loop controller such that global stability of the closed-loop system is guaranteed.
The design of the outer-loop uses components of a closed-loop reference model in a judicious manner which enables a modular approach, without any re-design of the inner-loop controller.
In addition, this architecture facilitates the use of an additional state-limiter to enforce desired limits on the state variables.

 A numerical example based on a scramjet powered, generic hypersonic vehicle model is presented, demonstrating the efficacy of the proposed control design.
 The six-degree-of-freedom nonlinear vehicle model is linearized, giving the design model for which the controller is synthesized.
 The adaptive output feedback controller is then applied to an evaluation model, which is nonlinear, coupled, and includes actuator dynamics, and it is shown to result in stable tracking in the presence of uncertainties that destabilize the baseline controller.
 Benefits of various aspects of the sequential and modular control design as well as its adaptive components are clearly illustrated in this numerical example.

\end{abstractpage}

\cleardoublepage%

\section*{Acknowledgments}

I would like to first thank my advisor Dr.\ Anuradha Annaswamy for her guidance throughout this research and during my time at MIT.\@
Despite the pressure that often comes from ones advisor, working with Dr.\ Annaswamy has, overall, been a great experience.
While she was often very demanding, she was also very supportive, and mentored me to ensure that my worked was not lacking in rigor.
She worked hard to make sure my funding was available so that I was able to continue focusing on research.
She provided many opportunities to gain experience, including spending the summers doing internships with my research sponsors, or presenting lectures for her class.
She was always available to meet, but also flexible to accommodate other aspects of life.
For this and more I am thankful to have been able to complete my Ph.D. under Dr.\ Annaswamy.

I would also like to thank Dr.\ Jonathan Muse and Dr.\ Michael Bolender in the Aerospace Systems Directorate at the Air Force Research Laboratory for providing the vehicle model for this research, giving their support and mentorship along the way, and providing direction during my early summer work at AFRL.\@
They have both always been very helpful and accessible, and worked hard to ensure that they were able to provide the funding to support me throughout my doctoral studies.
Our interactions on a bi-weekly basis were crucial in keeping me focused, and their close involvement with me throughout my studies was greatly appreciated.
These discussions were invaluable to me as a developing researcher.
They were always helpful in providing feedback on my work, making sure that what I developed over the last several years was not only mathematically rigorous, but also practical.

Thanks to Dr.\ Eugene Lavretsky of Boeing Research \& Technology for contributing his precious advice and suggestions throughout this research.
My meetings with Eugene were some of the most productive I have ever had.
During these meetings he was able, in just moments, to parse out the real challenges I was facing, figure out what was really going on, and provide valuable insights and suggestions.
Working with Eugene helped me to learn how to distill and communicate complicated ideas, and focus on what really matters.
He was always incredibly positive and supportive, great fun to interact with, and was constantly sharing interesting problems to think about and work on.

I'd like to thank the other members of my committee for their involvement as well.
I took classes with both Professor How and Professor Hogan during my first semester as a master's student at MIT in the fall of 2011.
These classes were some of my most memorable experiences at MIT, and the way they communicated the ideas and material has stuck with me ever since.
I am grateful for agreeing to serve on my committee, the time they have given me, and the feedback and suggestions they have provided.

Thanks also to all of my lab mates for making the past years in lab enjoyable, and making our classes together fun.
Their assistance and friendship over these past years has been great.
It's been a pleasure to share an office with Max Zheng Qu and Heather Hussain, and it was always nice to take a break from work from time-to-time to chat.
Our time outside of the office with Ben Jenkins was always fun as well, providing the brain a much needed break from school.

I also want to thank Dr.\ Melanie Lutz, without whom I would not be where I am today.
She was the catalyst that, after several years of my failed attempts to become a successful student, jump-started me onto my path towards MIT and a Ph.D.
The three classes I took with her at Solano Community College during the fall semester of 2008 have turned out to be the most life-changing classes I have taken.
She was the most amazing classroom professor I have ever had, making the learning process engaging, stimulating, and challenging.
Over the years, she constantly pushed me to be my best, never allowing me to give less.
I want to thank her for all of the encouragement, inspiration, guidance, help, and friendship over the past eight years, and for being a truly outstanding mentor.

I would like to thank all of my friends for their patience and understanding over the last several years as well.
I appreciate my friends' support and encouragement throughout this journey, and putting up with my unavailability for months on end while I was busy with coursework and research.
I'd like to thank Alessandro Babini, with whom I am embarking on life's next big challenge.
Thank you for keeping that moving forward while allowing me to remain focused on my studies, and complete my PhD.

Finally, I would like to thank my Mom, Dad, and all of my family for their unconditional love, support, and understanding all these years.
They have kept me positive throughout the demanding years as a student, and always helped to keep me going.
Thanks for everything!

This research is funded by the Air Force Research Laboratory/Aerospace Systems Directorate grant FA 8650--07--2--3744 for the Michigan/AFRL Collaborative Center in Control Sciences, with Dr.\ Michael Bolender as technical monitor, and the Universal Technology Corporation (UTC) contract FA8650--10--D--3037 subcontract agreement 15--S2606--04--C19 for Adaptive Flight Control for Hypersonic Vehicles with Integrated Loops.
