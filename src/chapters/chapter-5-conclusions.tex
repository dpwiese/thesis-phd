\chapter{Conclusions and Future Work}\label{ch.conclusions}

This thesis presented the design of an adaptive controller for a class of uncertain MIMO systems.
Synthesis of the controller is completed using a sequential loop closure based procedure, where an adaptive inner-loop controller is designed first, followed by an outer-loop controller.

The inner-loop controller is capable of accommodating the uncertainty present in the plant, only requires the plant output, and provides command tracking of an inner-loop regulated output.
The controller is uses a baseline Luneberger observer based controller, which, when combined with the adaptive element provides the closed-loop reference model structure.
This thesis provided a new way of synthesizing the gain matrices required for such a controller, providing a larger set of solutions and extra degrees of freedom to tune the controller for increased performance and robustness.

The outer-loop controller generates appropriate inner-loop commands so that the outer-loop plant regulated output follows a desired command trajectory.
The outer-loop controller incorporates a state-limiter, allowing the inner and outer-loop command signals to be modified as necessary to limit the evolution of the state trajectories to within a certain prescribed region within the state space.
The outer-loop controller also uses components of a closed-loop reference model, and the resulting closed-loop system is shown to be globally stable.

This sequential loop closure based procedure to synthesize an outer-loop controller simplifies the process of designing guidance and control laws from that of designing a single higher-order controller to several lower-order controllers.
The proposed approach provides an outer-loop control design which does not require a re-design of the existing inner-loop, and guarantees global stability of the closed-loop system, and enforces desired state limits.

Finally, this adaptive controller was applied to a highly nonlinear, unstable, six degree-of-freedom Generic Hypersonic Vehicle model, which includes unmodeled actuator dynamics.
The performance of the GHV was evaluated by providing altitude and heading commands, when subject to uncertainty.
The proposed adaptive controller was shown to provide stability when the baseline could not, accommodated the desired state limits, and provided command tracking.

The control design presented in this thesis has provided a new control architecture, and a constructive procedure to synthesize the required gain matrices to provide stability and command tracking.
This procedure used to synthesize the gain matrices has provided a large set of solutions, from which the control designer can select each of the desired matrices.
This has, in practice, provided very good control designs with little effort.
However, work remains to determine methods or rules by which to select from the set of possible controller solutions ones that are more desirable than others.
This may help to ensure the control gains are minimized, to provide specific frequency domain properties for the underlying baseline controller, or minimize control effort, for example.

In addition, the state limiter presented in Section~\ref{sec.outerLoop.stateLimiting} requires further analysis.
While some qualitative analysis was done to understand how the state limiter can be used to improve the time response of a system by enforcing limits, work remains to quantify this benefit more precisely.
In particular, the state limiter introduces many additional degrees of freedom that must be fully understood in order to more successfully design and implement  the limiter on future systems.
